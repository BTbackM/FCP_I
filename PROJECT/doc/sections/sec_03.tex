\section{Related Work}

  \subsection{Early Approaches}

    \subsubsection{Traditional computer vision techniques}

      Traditional computer vision techniques are based on hand-crafted features
      and shallow machine learning algorithms. They are not able to learn complex
      patterns and require a lot of human effort to design features. Therefore,
      they are not suitable for semantic segmentation. Some approaches include
      \ti{Edge detection}, \ti{Region growing}, \ti{Clustering}\cite{intelligence2021modern}.

    \subsubsection{Handcrafted features}

      SIFT and HOG are handcrafted features that are used for object detection
      and semantic segmentation. Although, they are not able to learn complex
      patterns and require a lot of human effort to design features\cite{goodfellow2016deep}.

  \subsection{Deep learning approaches}

    \subsubsection{Convolution factorization}

      \ti{Convolution factorization} is a technique that decomposes a standard
      convolution into a set of smaller convolutions. It is used to reduce the
      number of parameters and operations while maintaining the accuracy of the
      network\cite{mehta2018espnet}. Convolution factorization has shown being
      able to reduce computational complexity(i.e. Inception, ResNext, MobileNet).

    \subsubsection{Fully Convolutional Network\cite{long2015fully}}

      \ti{Fully Convolutional Network} (FCN) is a neural network architecture
      for semantic segmentation. It is based on a \ti{convolutional neural network}
      (CNN) with a modified architecture. The main idea is to replace the fully
      connected layers with convolutional layers. As it uses encoder-decoder architecture,
      encoder to extract features and to recover the spatial information through
      convolutional and pooling layers, and decoder uses transposed convolutional
      layers to upsample the feature maps to the original image size\cite{long2015fully}.

      However, FCN is computationally expensive and requires a lot of memory. As
      autonomous vehicles require real-time semantic segmentation and has limited
      computational resources, FCN is not suitable for this task.

    \subsubsection{Encoder-Decoder architecture: SegNet\cite{badrinarayanan2017segnet}}

      \ti{SegNet} is a neural network architecture for semantic segmentation. It
      is based on a \ti{convolutional neural network} (CNN) with a modified architecture.
      The main idea is to use a \ti{pooling indices} obtained from the max-pooling
      layer to perform up-sampling. This allows to reduce the number of parameters
      and operations while maintaining the accuracy of the network\cite{badrinarayanan2017segnet}.

      Even so, SegNet is computationally expensive and requires a lot of memory. As
      autonomous vehicles require real-time semantic segmentation and has limited
      computational resources, SegNet is not suitable for this task.

    \subsubsection{Pyramid Pooling Modules: PSPNet}

      \ti{PSPNet} uses \ti{Pyramid Pooling Modules} to capture the context in different
      regions of the image, with multiscale pooling layers to capture contextual
      information at different scales.
