\section{Introduction}

  Semantic segmentation, which involves classifying each pixel in an image
  into a set of predefined categories, is a fundamental task in computer vision,
  and has a crucial role in autonomous driving systems to enable the vehicle
  to understand and interpret its surroundings. Dividing the image into multiple
  semantic regions is a key step in the perception pipeline of autonomous vehicles,
  as it provides a rich representation of the environment identifying the instances
  of objects and their relations.

  % On the other hand, structured
  % understanding of the environment is a central ability in human cognition to
  % see the world as a collection of objects and their relations. It's a computationally
  % intensive task that requires deep learning models to process large amounts of
  % data in real-time.

  % NOTE: Model and hardware problems
  Autonomous vehicles operate in a wide range of environments, and thus, the real-time
  performance of the segmentation algorithm is critical as it's a key component required
  for safe and reliable decision-making. However, the computational resources available
  in autonomous vehicles are often limited due to physical constraints including memory capacity
  due to existence of large models and processing speed required for real-time responses.
  These limitations pose significant challenges for implementing efficient and accurate
  semantic segmentation algorithms on autonomous vehicles, as traditional methods typically
  rely on computationally expensive deep neural networks with many parameters.

  % NOTE: Dataset problems
  In addition, complex scenes, very often contain many instances and abstract
  levels, and high-resolution images to capture fine-grained details, pose
  additional challenges for semantic segmentation algorithms. \tb{Complex scenes} require
  the model to classify numerous objects accurately, even more
  if the model request some hierarchical information. However, achieving
  high accuracy in semantic segmentation is a challenging task due to presence of occlusions,
  object overlaps and lighting variations, which can lead to inaccurate classification. \tb{High-resolution}
  of the input images increase the computational demands of semantic segmentation models, which
  can lead to a significant drop in performance, making it difficult to achieve real-time processing
  required by autonomous vehicles.

  % NOTE: Proposed solution
  The objective is to enable real-time semantic segmentation on autonomous vehicles without
  compromising the model accuracy while keeping the computational requirements low. Also,
  should be able to handle different levels of scene detail and adapt to different driving
  scenarios, due to dynamic and changing environments.

  % Recently, transformers, a deep learning model have achieved state-of-the-art
  % performance in natural language processing, so the paper propose a novel hierarchical
  % semantic segmentation algorithm relying on transformers that's able to achieve high accuracy.

  To evaluate the proposed approach, made use of benchmark datasets(\emph{i.e.}
  Cityscapes\cite{cordts2016cityscapes}, Mapillary Vistas \cite{neuhold2017mapillary}) for semantic
  segmentation in autonomous driving scenarios. This paper findings have important implications for
  the development of \ti{practical} semantic segmentation algorithms for autonomous vehicles,
  deploying embedded systems with constrained computational resources, in order
  to contribute to the advancement of safe and reliable autonomous driving systems.
  % Code example
  % \begin{listing}[!ht]
  %   \inputminted[breaklines, fontfamily=SourceSansPro-TLF]{cpp}{../src/competitive_template.cpp}
  %   \caption{Testing \ti{competitive template} listing with \ti{minted}}
  %   \label{master_service}
  % \end{listing}
